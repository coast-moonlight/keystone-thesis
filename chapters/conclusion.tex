\chapter{Conclusion}
\label{chap:conclusion}

The research presented in this thesis has provided a comprehensive examination of the performance, scalability, and architectural constraints involved in deploying secure enclaves on RISC-V platforms through the Keystone Trusted Execution Environment framework. By conducting rigorous performance benchmarking, detailed microarchitectural analyses, and thorough system-level evaluations, this study has elucidated both the enabling strengths and the critical limitations inherent in current RISC-V TEE implementations, with particular focus on the Physical Memory Protection mechanism responsible for enclave isolation.

Through the empirical evaluation of a range of workloads—including compute-bound benchmarks and memory-intensive cryptographic algorithms—this thesis has demonstrated that enclave performance on RISC-V is highly workload-dependent. Compute-bound applications such as CoreMark and Dhrystone experience relatively minor performance degradation when executed within enclaves, typically ranging between 10 to 15 percent. This observation aligns with expectations given the dominance of predictable integer computations and limited memory operations in such workloads. In stark contrast, cryptographic workloads exemplified by the Kyber post-quantum algorithm suffered from substantial performance penalties exceeding 60 percent. This discrepancy arises from Kyber's memory-bound nature and its frequent indirect memory accesses, which exacerbate the overhead introduced by enclave memory isolation and associated integrity checks.

Beyond performance considerations, a critical bottleneck was identified in the utilization of PMP for enclave memory protection. The RISC-V architecture’s limit of sixteen PMP entries, each constrained by the Naturally Aligned Power-of-Two encoding scheme, imposes stringent limits on enclave scalability. Fragmented or non-contiguous memory allocations rapidly exhaust PMP entries, restricting the feasible number of concurrent enclaves. Empirical data confirmed that enclave instantiation failures occurred beyond eight simultaneous enclaves due to PMP saturation, a constraint that holds even on systems with abundant CPU cores and physical memory. This architectural limitation presents a significant barrier to multi-tenant or high-concurrency scenarios where isolated execution contexts must coexist.

The implications of these findings extend beyond immediate performance and capacity concerns. The current PMP-based isolation model, while offering lightweight hardware-enforced security guarantees, intrinsically restricts flexibility in enclave memory layout and scalability. Without modifications to either the hardware architecture—such as increasing PMP register counts or adopting more flexible memory region descriptors—or to system software managing memory allocation, the practicality of RISC-V enclaves in complex real-world applications remains limited.

To mitigate these challenges, this thesis advocates several best practices. Firstly, contiguous and properly aligned memory allocation is essential to minimize PMP entry consumption and reduce fragmentation. Secondly, consolidating multiple lightweight tasks into fewer enclaves can relieve PMP pressure. Thirdly, selectively isolating only the most security-critical components within enclaves, particularly in cryptographic workloads, balances security with performance. Furthermore, runtime systems could implement dynamic PMP entry management, reclaiming and reconfiguring PMP resources in response to enclave lifecycle events, though this approach demands additional synchronization mechanisms to preserve security guarantees. Lastly, optimizing enclave entry and exit frequency, particularly within tightly looped cryptographic operations, can help amortize synchronization costs.

Looking forward, this research underscores several promising directions. Architecturally, enhancements to RISC-V PMP—such as increasing the number of PMP entries or introducing hierarchical and flexible region descriptors—could substantially improve enclave scalability. From a software perspective, enclave-aware memory allocators and scheduling systems that proactively manage PMP resources and workload characteristics would enhance system robustness under high concurrency. Integration of cryptographic accelerators within the TEE boundary also holds promise to alleviate performance bottlenecks inherent in memory-bound cryptographic workloads. Compiler toolchains and enclave abstraction layers that enforce memory alignment and contiguity at build time could reduce developer burden and improve runtime efficiency.

In summary, this thesis has demonstrated that while RISC-V enclaves leveraging Keystone represent a compelling foundation for open and customizable trusted execution environments, practical deployment necessitates careful consideration of PMP-imposed constraints. Balancing security, performance, and scalability demands architectural awareness and judicious system design. The insights gained herein provide a detailed characterization of enclave behavior and highlight fundamental limitations, serving both as a guide for current practitioners and as a foundation for future research in secure computing on open hardware platforms. As trusted execution environments become increasingly vital across cloud, edge, and IoT domains, addressing these challenges will be critical to realizing scalable, efficient, and secure computing paradigms.

%6.1 Summary of Findings
%6.2 Limitations
%6.3 Directions for Future Research