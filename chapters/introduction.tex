\chapter{Introduction}
\label{chap:intro}
Modern computing increasingly relies on platforms that operate in potentially untrusted environments. These include public cloud infrastructures, multi-tenant servers, and distributed edge devices, where sensitive data and critical computations may be exposed to compromised software stacks or malicious actors. In such contexts, ensuring the confidentiality and integrity of both data and execution is a fundamental requirement. Conventional security mechanisms, such as access control and encryption, are often insufficient in scenarios where the underlying operating system, hypervisor, or even firmware may be compromised.

To address these challenges, Trusted Execution Environments (TEEs) have emerged as a compelling solution. TEEs provide hardware-assisted isolation mechanisms that enable the execution of code in a protected context, referred to as an enclave. This enclave is designed to be resilient to attacks originating from untrusted system components, including the operating system and other user-space applications. By establishing a secure boundary around sensitive code and data, TEEs allow developers to implement trustworthy services even on compromised hosts.

Several commercial implementations of TEEs exist today, with Intel Software Guard Extensions (SGX) and ARM TrustZone being the most prominent examples. These technologies have demonstrated the practical viability of hardware-enforced isolation, and they have been integrated into a variety of real-world applications, from digital rights management to secure machine learning inference. However, despite their utility, these solutions are inherently proprietary and closed-source. This lack of transparency imposes significant constraints on researchers, developers, and system architects who wish to investigate alternative TEE designs, customize security mechanisms, or perform formal verification. Moreover, the architectural rigidity of these platforms limits their adaptability to emerging use cases, especially in academic or experimental contexts.

To overcome these limitations, the Keystone framework introduces an open-source, modular TEE architecture built atop the RISC-V instruction set architecture (ISA). RISC-V itself is an open, extensible ISA that has gained significant traction in both academic and industrial settings. Keystone leverages RISC-V's flexibility to enable fine-grained control over enclave behavior and to support a minimal, verifiable trusted computing base (TCB). The framework allows researchers and developers to explore custom enclave policies, experiment with hardware-software co-design, and perform reproducible evaluations in a controlled setting.

This thesis presents a detailed investigation into the Keystone TEE framework. It explores Keystone's architectural foundations, its use of RISC-V hardware features such as Physical Memory Protection (PMP), and its support for dynamic enclave management. Special attention is paid to the performance implications of enclave isolation, as measured through controlled benchmarking in a virtualized environment. By evaluating Keystone under varying execution conditions, this work seeks to provide a rigorous assessment of its efficiency and suitability for secure computation on emerging RISC-V platforms.


\section{Contributions of this Thesis}

This thesis makes significant contributions to the study and evaluation of Trusted Execution Environments (TEEs), with a particular focus on the Keystone framework. The research encompasses architectural analysis, emulation-based deployment, and rigorous empirical performance evaluation.

Firstly, a comprehensive architectural analysis of the Keystone TEE framework is presented. This includes an in-depth discussion of its core components, such as the Security Monitor (SM), enclave runtime environment, and user-level application design. The role of RISC-V’s privilege hierarchy and Physical Memory Protection (PMP) in enforcing strict isolation guarantees is thoroughly elaborated, providing a holistic understanding of Keystone’s security architecture.

Secondly, the thesis conducts a rigorous comparative evaluation of Keystone alongside widely deployed commercial TEEs, specifically Intel Software Guard Extensions (SGX) and ARM TrustZone. This analysis highlights fundamental architectural distinctions, including enclave creation models, privilege separation strategies, scalability, Trusted Computing Base (TCB) size, and extensibility. Such comparison contextualizes Keystone’s design philosophy and underscores its advantages in openness and modularity.

Furthermore, the practical deployment of Keystone within a fully virtualized environment is demonstrated using a RISC-V-specific fork of the QEMU emulator. This approach facilitates development and testing independent of physical RISC-V hardware. The thesis also documents the toolchain configuration, build process, and debugging workflow, establishing a reproducible methodology for future research.

To quantify the runtime overhead associated with enclave-based execution, Keystone enclaves are evaluated using two established synthetic benchmarks: Dhrystone and CoreMark. Key performance metrics—including execution time, Dhrystones per second (DMIPS), iterations per second (IPS), and variability across multiple runs—are measured to assess the cost of enclave isolation and security enforcement.

Finally, the computational behavior of benchmark applications within the enclave context is analyzed to provide empirical insights into how TEE-induced isolation affects instruction throughput and memory access patterns. These findings illuminate the inherent trade-offs between performance and security in enclave-based execution models.

\section{Related Work}

The field of Trusted Execution Environments has been shaped by a combination of commercial deployments and academic proposals, each addressing different aspects of secure computing. This section situates the present work within the broader landscape of TEE research and implementation.

Intel SGX is one of the most well-known commercial TEE solutions. It introduces user-space enclaves with hardware-enforced isolation and memory encryption. SGX supports fine-grained protection mechanisms and remote attestation, making it suitable for confidential computing scenarios. However, SGX is hindered by several limitations, including rigid enclave memory constraints, lack of supervisor-mode support, and a closed-source implementation that precludes low-level architectural exploration or modification. These issues restrict its applicability in research and experimental systems development.

ARM TrustZone represents a different paradigm, based on the separation of execution into a Secure World and a Normal World. While TrustZone enables trusted operating systems and supports a variety of secure services, it suffers from scalability issues, particularly in multi-core environments where interrupt handling becomes a bottleneck. Moreover, TrustZone's coarse-grained isolation and reliance on a monolithic Secure World kernel contribute to a large TCB, which poses challenges for security assurance and formal verification.

In the academic domain, projects such as Sanctum and Sancus have contributed lightweight and formally verifiable secure execution frameworks. Sanctum builds on RISC-V to enforce secure memory isolation and secure multiplexing of shared hardware resources. Sancus targets embedded systems and emphasizes minimalism and provable security properties. While these efforts have made valuable theoretical contributions, they are often constrained by limited platform support and lack the general-purpose capabilities required for broader deployment or benchmarking.

Keystone addresses the limitations of both commercial and academic TEEs by adopting a fully open-source design built on RISC-V. It supports dynamic enclave creation, configurable runtime environments, and an extensible architecture suitable for incorporating advanced security primitives. Keystone’s modularity, support for supervisor-level enclave components, and compatibility with standard Linux-based host systems make it especially well-suited for research, prototyping, and educational use. This thesis builds upon this foundation by exploring the architectural, practical, and performance aspects of deploying Keystone in a virtualized environment, thereby contributing new empirical insights and documentation to the growing body of work on open TEEs.

\section{Structure of this Thesis}

This thesis is structured to progressively build an understanding of Trusted Execution Environments (TEEs) with a focus on the Keystone framework, moving from conceptual foundations to empirical evaluation and practical system design insights.

This thesis is organized as follows: Chapter~\ref{chap:intro} introduces the motivation for secure computation in untrusted environments, outlines the core contributions of this work, and surveys relevant literature in the domain of Trusted Execution Environments (TEEs). Chapter~\ref{chap:background} presents the technical foundations and system architecture of the Keystone framework, including a comparative analysis with Intel SGX and ARM TrustZone that highlights key design trade-offs and performance considerations. Chapter~\ref{chap:methodology} describes the experimental setup, detailing the virtualized testbed, benchmark selection, parameter tuning, and data collection methods employed to ensure rigorous and reproducible evaluation. Chapter~\ref{chap:results} reports empirical findings and analyzes the impact of system parameters such as CPU core count, memory allocation, and cache size on enclave performance, along with associated overheads and insights. Chapter~\ref{chap:recommendations} synthesizes these results into practical configuration guidelines for deploying Keystone TEEs across diverse workload scenarios and identifies architectural bottlenecks for future improvements. Finally, Chapter~\ref{chap:conclusion} summarizes the main contributions and discusses their implications for secure system design, outlining promising directions for further research including hardware extensions and formal verification.