\documentclass[english, version-2020-11]{uzl-thesis}
\UzLStyle{computer modern oldschool design}
\usepackage{comment}
\usepackage{float}

\UzLThesisSetup{
Masterarbeit,
Verfasst              = {am}{Institut für Technische Informatik},
Titel auf Deutsch     = {
    Bewertung von RISC-V-Enklaven: Leistungsbenchmarks und bewährte Konfigurationspraktiken
}, 
Titel auf Englisch    = {
    Evaluating RISC-V Enclaves: Performance Benchmarks and Configuration Best Practices 
},
Autor                 = {Basil Ugbomoiko},
Betreuerin            = {Dr.-Ing. Saleh Mulhem},
Mit Unterstützung von = {Henrik Strunck, M.Sc},
Studiengang           = {IT Security},
Datum                 = {31. August 2025},
Abstract = {This thesis evaluates the performance of Keystone enclaves, an open-source Trusted Execution Environment (TEE) designed for RISC-V platforms. Keystone enables secure computation by isolating sensitive workloads from the rest of the system. To assess its performance, the study benchmarks critical system parameters, including the number of CPU cores, cache size, memory allocation, and enclave configuration. Using industry-standard benchmarking tools such as Dhrystone and CoreMark, the research measures execution time, context switch overhead, memory latency, and CPU utilization across a range of hardware and software configurations. The experimental analysis identifies key performance bottlenecks that impact the efficiency of enclave execution. Based on these findings, the study presents configuration best practices and tuning recommendations tailored to various workload profiles, such as compute-bound, memory-intensive, and mixed applications. The insights gained from this evaluation contribute to optimizing the deployment of Keystone enclaves, making them more viable for real-world use cases that demand secure and efficient execution on RISC-V architectures.},
Zusammenfassung = {Diese Arbeit bewertet die Leistung von Keystone-Enklaven, einer Open-Source Trusted Execution Environment (TEE) für RISC-V-Plattformen. Keystone ermöglicht sichere Berechnungen, indem sensible Workloads vom restlichen System isoliert werden. Zur Leistungsbewertung werden zentrale Systemparameter wie die Anzahl der CPU-Kerne, die Cache-Größe, die Speicherzuweisung und die Konfiguration der Enklave untersucht. Mithilfe standardisierter Benchmarking-Tools wie Dhrystone und CoreMark werden Ausführungszeit, Kontextwechsel-Overhead, Speicherlatenz und CPU-Auslastung unter verschiedenen Hardware- und Softwarekonfigurationen gemessen. Die experimentelle Analyse identifiziert wichtige Leistungsengpässe, die die Effizienz der Enklaven beeinträchtigen. Auf Basis dieser Erkenntnisse werden Konfigurationsrichtlinien und Optimierungsempfehlungen vorgestellt, die auf unterschiedliche Workload-Profile zugeschnitten sind, z. B. rechenintensive, speicherintensive und gemischte Anwendungen. Die gewonnenen Erkenntnisse tragen dazu bei, den Einsatz von Keystone-Enklaven zu optimieren und ihre praktische Anwendbarkeit in sicherheitskritischen Bereichen auf RISC-V-Architekturen zu verbessern.},
Numerische Bibliographie
} % Put your \UzLStyle and \UzLThesisSetup here

\begin{document}

\chapter{Introduction}

Modern computing increasingly relies on platforms that operate in potentially untrusted environments. These include public cloud infrastructures, multi-tenant servers, and distributed edge devices, where sensitive data and critical computations may be exposed to compromised software stacks or malicious actors. In such contexts, ensuring the confidentiality and integrity of both data and execution is a fundamental requirement. Conventional security mechanisms, such as access control and encryption, are often insufficient in scenarios where the underlying operating system, hypervisor, or even firmware may be compromised.

To address these challenges, Trusted Execution Environments (TEEs) have emerged as a compelling solution. TEEs provide hardware-assisted isolation mechanisms that enable the execution of code in a protected context, referred to as an enclave. This enclave is designed to be resilient to attacks originating from untrusted system components, including the operating system and other user-space applications. By establishing a secure boundary around sensitive code and data, TEEs allow developers to implement trustworthy services even on compromised hosts.

Several commercial implementations of TEEs exist today, with Intel Software Guard Extensions (SGX) and ARM TrustZone being the most prominent examples. These technologies have demonstrated the practical viability of hardware-enforced isolation, and they have been integrated into a variety of real-world applications, from digital rights management to secure machine learning inference. However, despite their utility, these solutions are inherently proprietary and closed-source. This lack of transparency imposes significant constraints on researchers, developers, and system architects who wish to investigate alternative TEE designs, customize security mechanisms, or perform formal verification. Moreover, the architectural rigidity of these platforms limits their adaptability to emerging use cases, especially in academic or experimental contexts.

To overcome these limitations, the Keystone framework introduces an open-source, modular TEE architecture built atop the RISC-V instruction set architecture (ISA). RISC-V itself is an open, extensible ISA that has gained significant traction in both academic and industrial settings. Keystone leverages RISC-V's flexibility to enable fine-grained control over enclave behavior and to support a minimal, verifiable trusted computing base (TCB). The framework allows researchers and developers to explore custom enclave policies, experiment with hardware-software co-design, and perform reproducible evaluations in a controlled setting.

This thesis presents a detailed investigation into the Keystone TEE framework. It explores Keystone's architectural foundations, its use of RISC-V hardware features such as Physical Memory Protection (PMP), and its support for dynamic enclave management. Special attention is paid to the performance implications of enclave isolation, as measured through controlled benchmarking in a virtualized environment. By evaluating Keystone under varying execution conditions, this work seeks to provide a rigorous assessment of its efficiency and suitability for secure computation on emerging RISC-V platforms.
\section{Contributions of this Thesis}
This thesis makes several substantive contributions to the study and evaluation of Trusted Execution Environments (TEEs), with a particular emphasis on the Keystone framework. The research spans architectural analysis, emulation-based deployment, and empirical performance evaluation. The primary contributions are enumerated below:

\begin{itemize}
\item \textbf{Comprehensive Architectural Analysis of Keystone:}
The thesis presents a detailed and systematic examination of the Keystone TEE framework. This includes an in-depth discussion of its core architectural elements, such as the Security Monitor (SM), enclave runtime environment, and user-level application design. The role of RISC-V’s privilege hierarchy and Physical Memory Protection (PMP) in enforcing strict isolation guarantees is elaborated, thereby providing a holistic view of Keystone’s security architecture.
\item \textbf{Comparative Evaluation with Established TEEs:}  
A comparative study is conducted to evaluate Keystone alongside widely deployed commercial TEEs, namely Intel Software Guard Extensions (SGX) and ARM TrustZone. This analysis investigates fundamental architectural distinctions, including enclave creation models, privilege separation strategies, scalability, TCB size, and extensibility. The comparison helps contextualize Keystone’s design philosophy and clarifies its advantages in terms of openness and modularity.

\item \textbf{Virtualized Deployment and Toolchain Integration:}  
The thesis demonstrates the practical deployment of Keystone in a fully virtualized environment using a RISC-V-specific fork of the QEMU emulator. This setup facilitates development and testing without the need for physical RISC-V hardware. It also documents the toolchain configuration, build process, and debugging workflow, thereby offering a reproducible methodology for future research and evaluation.

\item \textbf{Performance Evaluation Using Synthetic Benchmarking:}  
To assess the runtime impact of enclave-based execution, Keystone enclaves are benchmarked using two widely accepted synthetic workloads: Dhrystone and CoreMark. The thesis measures key performance indicators—including execution time, Dhrystones per second (DMIPS), iterations per second (IPS), and standard deviation across runs—to quantify the overhead introduced by enclave isolation and security enforcement.

\item \textbf{Workload Characterization Under Enclave Constraints:}  
The computational behavior of the benchmark applications is analyzed in the context of enclave execution. The findings provide empirical insight into how TEE-induced isolation affects instruction throughput and memory interaction patterns, thus illustrating the performance-security trade-offs inherent to enclave-based models.
\end{itemize}

\section{Related Work}

The field of Trusted Execution Environments has been shaped by a combination of commercial deployments and academic proposals, each addressing different aspects of secure computing. This section situates the present work within the broader landscape of TEE research and implementation.

Intel SGX is one of the most well-known commercial TEE solutions. It introduces user-space enclaves with hardware-enforced isolation and memory encryption. SGX supports fine-grained protection mechanisms and remote attestation, making it suitable for confidential computing scenarios. However, SGX is hindered by several limitations, including rigid enclave memory constraints, lack of supervisor-mode support, and a closed-source implementation that precludes low-level architectural exploration or modification. These issues restrict its applicability in research and experimental systems development.

ARM TrustZone represents a different paradigm, based on the separation of execution into a Secure World and a Normal World. While TrustZone enables trusted operating systems and supports a variety of secure services, it suffers from scalability issues, particularly in multi-core environments where interrupt handling becomes a bottleneck. Moreover, TrustZone's coarse-grained isolation and reliance on a monolithic Secure World kernel contribute to a large TCB, which poses challenges for security assurance and formal verification.

In the academic domain, projects such as Sanctum and Sancus have contributed lightweight and formally verifiable secure execution frameworks. Sanctum builds on RISC-V to enforce secure memory isolation and secure multiplexing of shared hardware resources. Sancus targets embedded systems and emphasizes minimalism and provable security properties. While these efforts have made valuable theoretical contributions, they are often constrained by limited platform support and lack the general-purpose capabilities required for broader deployment or benchmarking.

Keystone addresses the limitations of both commercial and academic TEEs by adopting a fully open-source design built on RISC-V. It supports dynamic enclave creation, configurable runtime environments, and an extensible architecture suitable for incorporating advanced security primitives. Keystone’s modularity, support for supervisor-level enclave components, and compatibility with standard Linux-based host systems make it especially well-suited for research, prototyping, and educational use. This thesis builds upon this foundation by exploring the architectural, practical, and performance aspects of deploying Keystone in a virtualized environment, thereby contributing new empirical insights and documentation to the growing body of work on open TEEs.

\section{Structure of this Thesis}

This thesis is structured to progressively build an understanding of Trusted Execution Environments (TEEs) with a focus on the Keystone framework, moving from conceptual foundations to empirical evaluation and practical system design insights.

\begin{itemize}
    \item \textbf{Chapter 1: Introduction} introduces the motivation for secure computation in untrusted environments, outlines the core contributions of this thesis, and reviews relevant related work in the domain of TEEs.
    
    \item \textbf{Chapter 2: Background and System Architecture} provides technical foundations for understanding TEEs, focusing on the design of the Keystone framework. It also presents a comparative analysis with Intel SGX and ARM TrustZone, highlighting design trade-offs and performance considerations.
    
    \item \textbf{Chapter 3: Methodology} outlines the experimental strategy used to evaluate Keystone. It describes the virtualized testbed, benchmark selection, parameter tuning, and data collection procedures used to ensure fair and reproducible performance measurements.
    
    \item \textbf{Chapter 4: Results and Discussion} presents empirical findings and analyzes the impact of system parameters such as CPU core count, memory allocation, and cache size on enclave performance. It also discusses performance overheads and key insights derived from the results.
    
    \item \textbf{Chapter 5: System Configuration Recommendations} synthesizes the experimental results into actionable guidelines for deploying Keystone TEEs in different workload scenarios. It identifies architectural bottlenecks and suggests improvements for future system designs.
    
    \item \textbf{Chapter 6: Conclusion} summarizes the contributions and outcomes of the thesis, reflecting on their implications for secure system design. It also outlines directions for future research, including hardware extensions and formal verification of trusted components.
\end{itemize}

\chapter{Background and System Architecture}
The Keystone framework is built on the RISC-V instruction set architecture, which provides a unique foundation for secure system design due to its open specification and modular extensibility. Unlike traditional ISAs that are controlled by single vendors, RISC-V encourages community-driven enhancements and supports custom extensions, making it particularly attractive for trusted computing research and development. Its clean and well-documented privilege model allows for the implementation of isolation mechanisms that are critical to the construction of TEEs.

RISC-V defines three primary privilege levels: user mode (U-mode), supervisor mode (S-mode), and machine mode (M-mode). These privilege levels are hierarchical, with M-mode having the highest authority over system resources. M-mode has direct access to all hardware features and is responsible for critical functions such as interrupt control, exception handling, and access to physical memory protection mechanisms. Keystone leverages this privilege model by placing its Security Monitor (SM)—the core of the TCB—in M-mode. The SM is responsible for managing enclaves, enforcing memory isolation, and mediating access to sensitive system operations. By situating the TCB in M-mode, Keystone ensures that enclave isolation policies are enforced at the highest level of privilege, thereby minimizing the attack surface.

A central hardware feature used by Keystone is Physical Memory Protection (PMP), introduced in version 1.10 of the RISC-V Privilege Specification. PMP enables M-mode software to define access permissions for specific physical memory regions, controlling which lower-privilege modes (U and S) may read, write, or execute code in those regions. This capability forms the basis of Keystone’s memory isolation model. Each enclave is allocated a private memory region protected by PMP, ensuring that neither the host operating system nor any unauthorized software can access its contents. PMP rules are enforced directly by hardware, providing strong guarantees of isolation that are resistant to software-based attacks.

In addition to memory protection, RISC-V supports a flexible trap and exception handling mechanism. M-mode can intercept and handle all traps, but may delegate certain classes of exceptions—such as system calls and page faults—to S-mode for performance optimization. This delegation allows the enclave runtime to cooperate with the host operating system for memory management tasks, while still maintaining strict control over enclave boundaries. The virtual memory subsystem is managed through the standard RISC-V MMU, which translates virtual addresses to physical addresses using multi-level page tables. Keystone leverages this infrastructure to support enclaves that operate with their own private address spaces, protected by PMP from tampering or inspection.

Importantly, the use of standard programming models and toolchains for M-mode software development makes the Keystone Security Monitor both maintainable and extensible. Unlike microcode or hardwired logic, which are difficult to audit and update, the SM is implemented in C and can be subjected to conventional testing and formal verification techniques. This increases the transparency and trustworthiness of the TCB, which is a critical requirement for secure system design.

Taken together, the combination of RISC-V’s privilege hierarchy, PMP, trap delegation mechanisms, and extensibility provides a solid foundation for implementing robust and configurable TEEs. The Keystone framework capitalizes on these features to deliver a flexible and open platform for secure enclave execution, while remaining faithful to the RISC-V philosophy of openness and modularity.

\section{Trusted Execution Environments}
Trusted Execution Environments (TEEs) are isolated computing environments that provide strong security guarantees for code and data execution, even in the presence of potentially compromised operating systems and applications. They achieve this by leveraging hardware-based isolation mechanisms to establish secure boundaries, often called enclaves, where sensitive computation can occur without interference from the rest of the system. TEEs aim to ensure confidentiality and integrity through careful hardware and software co-design, minimizing the trusted computing base (TCB) and mitigating a wide range of software attacks. While several commercial TEEs, such as Intel Software Guard Extensions (SGX) and ARM TrustZone, demonstrate the feasibility and practical relevance of enclave-based secure execution, their closed-source and inflexible designs limit research and hardware/software co-design opportunities. This has motivated the development of open TEE frameworks like Keystone, built on RISC-V—an open and extensible instruction set architecture. A key architectural feature of TEEs is privilege separation, where a minimal TCB executes in a highly privileged mode controlling access to critical physical resources, while the operating system and applications run in less privileged, untrusted modes.

\section{Keystone architecture overview}

Keystone represents an open-source framework designed to facilitate the architecture and implementation of Trusted Execution Environments (TEEs) leveraging hardware enclaves on RISC-V processors. Its primary objective is to enable the construction of customizable TEEs optimized for specific RISC-V platforms, thereby achieving enhanced performance, reduced trusted computing base (TCB) complexity, and improved programmability tailored to distinct workloads and security threat models.

A typical Keystone-capable system architecture comprises multiple components distributed across different privilege modes of the RISC-V privilege hierarchy. The foundational hardware is a trusted CPU package integrating Keystone-compatible RISC-V cores and a silicon root of trust, potentially augmented by features such as cache partitioning, memory encryption, and secure randomness sources. The core security functionality is governed by the Security Monitor (SM), a minimalistic M-mode software component embodying the system's TCB. The SM is responsible for enclave lifecycle management and enforces isolation boundaries between enclaves and the untrusted operating system (OS).

Keystone’s Security Monitor (SM), executing at the highest privilege level (M-mode), serves as the linchpin of the system’s trusted computing base. Beyond enclave lifecycle management, the SM enforces fine-grained memory isolation policies and orchestrates enclave scheduling, maintaining strict control over resource access to prevent unauthorized interference.

\begin{figure}[htbp]
    \centering
    \includegraphics[width=0.9\linewidth]{figures/keystone_overview.png}
    \caption{Overview of the Keystone architecture illustrating components such as the Security Monitor, enclave runtime, and the privilege hierarchy.}
    \label{fig:keystone_overview}
\end{figure}

Enclaves, the fundamental isolation units in Keystone, operate within dedicated physical memory regions inaccessible to the OS and other enclaves. Physical Memory Protection (PMP), a hardware-assisted mechanism controlled by the SM, restricts access to enclave memory exclusively to the enclave and the SM, thus preserving confidentiality and integrity even in the event of OS compromise.

Each enclave comprises two principal layers: the user-level enclave application (eapp) and a supervisor-level runtime environment. The eapp executes user-specific logic within the enclave, while the runtime, operating in supervisor mode (S-mode), manages system calls, exception handling, and virtual memory services intrinsic to enclave operation. This layered design provides a clear separation of concerns, reducing attack surfaces and allowing for customized security policies adapted to workload requirements.

Keystone’s workflow delineates distinct roles for platform providers and enclave developers, fostering modularity and flexibility. Platform providers undertake the compilation and deployment of the SM tailored to the target hardware, ensuring integration of the root of trust and hardware-specific functionalities. Enclave developers leverage the Keystone SDK to construct enclave applications alongside their runtimes and host binaries, which are subsequently packaged and deployed on the target platform independently of underlying hardware specifics.

Furthermore, Keystone supports remote attestation mechanisms, enabling verification of enclave authenticity and integrity prior to provisioning sensitive data or executing critical workloads. This capability is essential for secure deployment in distributed and cloud environments.

The enclave lifecycle progresses through three stages: creation, execution, and destruction. Creation involves the host allocating enclave private memory (EPM), populating it with enclave page tables, runtime, and application binaries, and invoking the SM to isolate the enclave via PMP. During execution, the SM orchestrates transitions into and out of the enclave, dynamically adjusting PMP permissions to maintain strict isolation. Destruction securely clears the enclave’s EPM and reclaims resources, ensuring no residual data remains.

\begin{figure}[htbp]
    \centering
    \includegraphics[width=0.9\linewidth]{figures/enclave_lifecycle.png}
    \caption{Stages of a Keystone enclave lifecycle: creation, execution, and destruction.}
    \label{fig:enclave_lifecycle}
\end{figure}

Keystone’s design also facilitates extensibility, allowing for integration of advanced security features such as secure I/O, cryptographic accelerators, and hardware-assisted debugging. This adaptability future-proofs the architecture, enabling continuous evolution of TEE capabilities aligned with emerging threats and diverse application demands on RISC-V platforms.

\section{Comparison with other TEEs}

Keystone distinguishes itself from widely adopted commercial Trusted Execution Environments (TEEs) such as ARM TrustZone and Intel Software Guard Extensions (SGX) through its open-source design, extensibility, and architectural modularity. While TrustZone and SGX have demonstrated practical value in securing sensitive computations on commodity hardware, their rigid implementations impose significant limitations on research, customization, and fine-grained hardware/software co-design.

ARM TrustZone implements a two-world model in which the processor switches between a Secure World (TEE) and a Normal World (REE) via secure monitor calls (SMCs). Each world has its own kernel and supports user- and supervisor-level execution. Trusted operating systems such as OP-TEE run in the Secure World and conform to the GlobalPlatform TEE Internal API for developing trusted applications. However, the TrustZone model provisions only a single, statically allocated TEE at boot time, limiting dynamic enclave creation and scalability. Additionally, most interrupts are directed to the Normal World, requiring costly context switches and complicating secure interrupt handling. These constraints, coupled with a relatively large trusted computing base (TCB), hinder the flexibility and security assurances necessary for diverse application scenarios.

Intel SGX adopts a contrasting enclave-based model where enclaves are dynamically created memory regions within user-space processes. These enclaves execute at user level (ring 3) and are protected by hardware-enforced memory encryption. SGX supports secure communication between the host application and the enclave through ECALL and OCALL interfaces, generated and verified using the edger8r tool. While this architecture allows finer-grained protection of sensitive data and logic, SGX lacks supervisor-mode support within enclaves and is tightly coupled to Intel's proprietary hardware platform. Furthermore, enclave memory is pre-allocated at system startup with strict size limitations (e.g., 128MB for SGX v1), restricting its scalability for memory-intensive secure workloads.

Keystone combines and extends the architectural principles of both TrustZone and SGX by enabling dynamic enclave creation, hierarchical privilege separation, and a minimal TCB architecture. Built on the RISC-V instruction set architecture, Keystone features a dedicated Security Monitor (SM) executing in machine mode (M-mode), which governs enclave lifecycle management and access control. Each enclave consists of a user-level application (eapp) and a supervisor-level runtime, allowing greater operational flexibility within the enclave itself. Memory for enclaves is dynamically provisioned by the untrusted operating system and isolated using the RISC-V Physical Memory Protection (PMP) mechanism, ensuring strong spatial and temporal isolation.

To enhance portability and interoperability, Keystone supports a partial implementation of the GlobalPlatform TEE Internal API and introduces keyedger, a secure communication interface similar to SGX's edger8r. Unlike SGX, however, Keystone is designed to be extensible, enabling the integration of advanced security primitives such as secure I/O, remote attestation, and custom scheduling policies. Its open-source nature and hardware-agnostic design position Keystone as a research-friendly and customizable TEE platform, suitable for both academic exploration and real-world deployment scenarios requiring secure, flexible, and verifiable execution environments.


\section{Performance considerations in TEEs}

\chapter{Methodology}
The primary objective of this thesis is to assess the performance impact of Keystone’s enclave isolation mechanisms on typical embedded and systems-level workloads. Specifically, the study seeks to quantify the computational overhead introduced by executing applications within a Keystone enclave, as compared to their execution in a conventional, non-isolated environment. To achieve this, a series of carefully controlled experiments were designed and executed within a virtualized RISC-V system based on QEMU.

All benchmarks were executed in an emulated environment configured for the RV64GC architecture, which is representative of 64-bit RISC-V systems targeted by Keystone. The use of QEMU for emulation allowed for fine-grained control over the system environment, ensured repeatability, and eliminated the variability introduced by physical hardware, such as thermal throttling or hardware-specific optimizations. Although virtualized, the emulation accurately models instruction execution, privilege transitions, memory access patterns, and I/O behavior, making it suitable for preliminary performance evaluation.

To evaluate the impact of enclave execution, each benchmark was implemented as a standalone enclave application (referred to as an “eapp”), accompanied by a corresponding host application that manages enclave lifecycle events. The host application, running in user space, is responsible for initializing the enclave, loading the benchmark binary, invoking enclave entry points, and handling edge calls used for communication between the enclave and the host. The enclave application, in contrast, is isolated by the Keystone runtime and contains only the benchmark logic and internal data structures. This modular separation ensures that the same benchmark code can be executed both natively (without an enclave) and securely (within an enclave) without structural modification, enabling an accurate comparative analysis.

The evaluation involved executing each benchmark—Dhrystone and CoreMark—under two distinct configurations:

\begin{enumerate}
\item \textit{Native Execution:} The benchmark runs as a conventional user-space process within the QEMU-emulated Linux system, without invoking Keystone enclave services.
\item \textit{Enclave Execution:} The benchmark is loaded into a Keystone enclave, with isolation enforced by the Security Monitor using PMP.
\end{enumerate}

Each benchmark was run multiple times (typically ten iterations per configuration) to account for performance variability and ensure statistical robustness. For each run, metrics such as total execution time, throughput (measured in DMIPS for Dhrystone and iterations per second for CoreMark), and standard deviation were collected. The comparative analysis focused on the relative performance degradation observed in the enclave configuration, thereby providing a direct measurement of the cost of security.

This methodology enables a nuanced understanding of how Keystone’s isolation features influence real-world performance metrics. The controlled nature of the test environment, coupled with repeated measurement and the use of industry-standard benchmarks, ensures that the results are both meaningful and reproducible. By isolating the effect of enclave execution, this study contributes valuable insights into the trade-offs between security and efficiency in open-source TEE architectures.
\section{Experimental setup}
%The experimental setup utilizes the Keystone-enhanced QEMU emulator running on an Ubuntu host operating system. This configuration allows for flexible development and testing of Keystone enclaves in a controlled environment without requiring physical RISC-V hardware. The Ubuntu host provides a stable platform for building and deploying enclave binaries, and it facilitates the integration of the Keystone kernel driver and SDK.

The Keystone framework was evaluated using a RISC-V specific fork of the QEMU emulator on an Ubuntu 22.04 host system. This configuration allowed testing of Keystone enclaves in a controlled virtualized environment, avoiding the need for physical RISC-V hardware. The system was configured to emulate the RV64 architecture for evaluating Keystone’s performance.

The build process for Keystone utilized the Make and Buildroot tools. Key configuration options included the selection of the QEMU virtual platform, the RV64 architecture, and the corresponding Buildroot configuration. Keystone was initiated in the QEMU environment with the following command:

\begin{verbatim}
make run
\end{verbatim}

This command launched the Security Monitor and the Linux OS. Within the QEMU guest, the Keystone kernel driver was manually loaded using:

\begin{verbatim}
modprobe keystone-driver
\end{verbatim}

For debugging, Keystone was run in debug mode with the QEMU GDB server connected to inspect memory protection and control registers:

\begin{verbatim}
KEYSTONE_DEBUG=y make run
make debug-connect
\end{verbatim}

Additional details on the build process, dependencies, and configuration options are available in the \textbf{Appendix}.

\section{Benchmarking tools and metrics}

The performance characterization of Keystone utilized two widely adopted synthetic benchmarks: \textit{Dhrystone} and \textit{CoreMark}. Both benchmarks are designed for embedded system evaluation and are well-supported in RISC-V environments. They were chosen to capture the computational characteristics most relevant to TEE deployment, including processor throughput, control flow, and moderate memory usage.

\textit{Dhrystone} primarily evaluates integer arithmetic and control flow performance, making it suitable for measuring core CPU throughput in environments where floating-point operations and I/O are secondary. The RISC-V implementation, sourced from the \texttt{riscv-tests} repository, was used to compute \textit{Dhrystones per second (DMIPS)}. Measurements were taken both in native execution and within a Keystone enclave to quantify the performance overhead introduced by enclave isolation.

\textit{CoreMark}, developed by EEMBC, offers a broader performance profile by incorporating common algorithmic workloads such as list processing, matrix manipulation, and finite-state machine evaluation. While still CPU-bound, CoreMark introduces moderate memory access patterns through its internal data structures. The benchmark was obtained from the SiFive repository and used to calculate \textit{iterations per second (IPS)}.

Performance metrics collected for both benchmarks included:
\begin{itemize}
    \item \textbf{Execution Time:} This is the total elapsed wall-clock time taken for a benchmark to complete execution. It serves as the most direct measure of performance, indicating how long the CPU and memory resources are engaged. Comparing execution times between native and enclave runs highlights the overhead introduced by enclave isolation and security mechanisms.

    \item \textbf{Dhrystones per Second (DMIPS):} Derived from the Dhrystone benchmark, DMIPS represents a normalized measure of integer computing throughput. It quantifies how many Dhrystone operations the CPU can perform per second, serving as a standardized metric for CPU integer performance. A reduction in DMIPS inside the enclave context signals the performance cost of enclave execution.

    \item \textbf{Iterations per Second (IPS):} Obtained from the CoreMark benchmark, IPS measures the number of complete benchmark iterations executed per second. Since CoreMark involves a mixture of algorithmic workloads and memory operations, IPS reflects a broader system performance indicator, capturing both CPU and memory subsystem efficiency under enclave constraints.

    \item \textbf{Standard Deviation:} Calculated across multiple repeated runs of each benchmark, the standard deviation quantifies the variability or consistency of the measured execution times and throughput values. Low standard deviation indicates stable and repeatable performance, while higher values may reveal sensitivity to transient system conditions or measurement noise.

\end{itemize}

\noindent\textbf{Benchmark Characteristics:}
\begin{itemize}
    \item \textit{CPU Intensive:} Both Dhrystone and CoreMark are designed to stress processor logic and control flow. Dhrystone focuses exclusively on integer arithmetic, while CoreMark includes a mix of computational tasks without relying on floating-point operations. These properties make them ideal for evaluating CPU performance in TEE contexts.
    
    \item \textit{Memory Interaction:} Although not memory-intensive by design, CoreMark exercises moderate memory usage through data structure manipulation, which may expose latency differences due to enclave-related memory isolation. Dhrystone has minimal memory interaction and serves as a purer test of CPU throughput.

    \item \textit{I/O and Storage:} Neither benchmark performs file or persistent storage operations. As such, storage-related TEE APIs (e.g., secure storage or object I/O) are not exercised and are outside the scope of this performance evaluation.
\end{itemize}

This dual-benchmark approach ensures a balanced evaluation of Keystone’s impact on both low-level computational performance and higher-level program behavior in a secure enclave environment.

\section{Parameter Variation Strategy}


\section{Data Collection and Analysis Procedures}

\chapter{Results and discussion}
\chapter{Experimental Results and Analysis}
\label{chap:results}

\section{Baseline Performance of Native Execution}

\section{Enclave Performance: Overheads and Latency}

\section{Kyber Performance in the Enclave vs Native Execution}

\section{Resource Utilization: Native vs Enclave}

\section{Cross-Workload Comparison of Enclave and Native Execution}

\section{Performance Comparison: Keystone vs Intel SGX/ARM TrustZone}

\chapter{System Configuration Recommendations}
\chapter{System Configuration Recommendations}
\label{chap:recommendations}

\section{General Configuration Guidelines}

\section{Recommendations for Compute-intensive Workloads}

\section{Recommendations for Memory-intensive Workloads}

\section{Limitations and Bottlenecks}

\section{Implications for Future TEE Design}

\chapter{Conclusion}
%\chapter{Conclusion}
\label{chap:conclusion}

%6.1 Summary of Findings
%6.2 Limitations
%6.3 Directions for Future Research

\begin{bibtex-entries}
@TechReport{Kernighan1974,
author = {Brian Kernighan},
title = {Programming in C – A Tutorial},
institution = {Bell Laboratories},
year = {1974}
}

@article{suzaki2021ts,
  title={Ts-perf: General performance measurement of trusted execution environment and rich execution environment on intel sgx, arm trustzone, and risc-v keystone},
  author={Suzaki, Kuniyasu and Nakajima, Kenta and Oi, Tsukasa and Tsukamoto, Akira},
  journal={IEEE Access},
  volume={9},
  pages={133520--133530},
  year={2021},
  publisher={IEEE}
}

\end{bibtex-entries}
\end{document}