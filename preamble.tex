\UzLThesisSetup{
Masterarbeit,
Logo-Dateiname        = {uzl-iti-logo-uzl.pdf},
Verfasst              = {am}{Institut für Technische Informatik},
Titel auf Deutsch     = {
    Bewertung von RISC-V-Enklaven: Leistungsbenchmarks und bewährte Konfigurationspraktiken
}, 
Titel auf Englisch    = {
    Evaluating RISC-V Enclaves: Performance Benchmarks and Configuration Best Practices 
},
Autor                 = {Basil Ugbomoiko},
Betreuerin            = {Dr.-Ing. Saleh Mulhem},
Mit Unterstützung von = {Henrik Strunck, M.Sc.},
Studiengang           = {IT Security},
Datum                 = {31. August 2025},
Abstract = {
  This thesis evaluates the performance of Keystone enclaves, an open-source Trusted Execution Environment (TEE) for RISC-V platforms. Keystone isolates sensitive workloads to enable secure computation, and its performance is assessed across CPU cores, cache sizes, memory allocations, and enclave configurations. Using Dhrystone, CoreMark, and cryptographic workloads such as Kyber—a post-quantum key encapsulation mechanism—this work measures execution time, context switch overhead, memory latency, and CPU utilization. The results reveal critical bottlenecks and provide best-practice configuration guidelines for compute-bound, memory-intensive, mixed, and post-quantum workloads, supporting efficient and secure deployment of Keystone enclaves on RISC-V architectures.
},
Zusammenfassung = {
  Diese Arbeit bewertet die Leistung von Keystone-Enklaven, einer Open-Source Trusted Execution Environment (TEE) für RISC-V-Plattformen. Keystone isoliert sensible Workloads zur sicheren Berechnung, wobei Leistungseinflüsse durch CPU-Kerne, Cache-Größe, Speicherzuweisung und Enklavenkonfiguration untersucht werden. Mithilfe von Dhrystone, CoreMark und kryptografischen Workloads wie Kyber—einem postquantischen Schlüssel-Kapselungsverfahren—werden Ausführungszeit, Kontextwechsel-Overhead, Speicherlatenz und CPU-Auslastung gemessen. Die Ergebnisse zeigen zentrale Engpässe auf und liefern Konfigurationsrichtlinien für rechenintensive, speicherintensive, gemischte und postquantische Workloads, um Keystone-Enklaven effizient und sicher auf RISC-V-Architekturen einzusetzen.
},
Acknowledgements = {
    I would like to sincerely thank Daniel K. for his insightful guidance and clarifications, which added immense value to this work.
    I am also grateful to Tom K. for providing the initial log analysis script and for his valuable input on effective visualization approaches.
    My heartfelt thanks go to my parents for their patience, encouragement, and the opportunities they have provided.  
    I deeply appreciate my supervisors, Dr.-Ing. Saleh M. and Henrik S. (M.Sc.), for their guidance, constructive feedback, and steadfast support throughout this project.  
    Finally, I extend my gratitude to my fellow Lübeckers and all others who contributed in various ways; their support, though too numerous to mention individually, was invaluable.
},
Numerische Bibliographie
}