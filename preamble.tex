\UzLThesisSetup{
Masterarbeit,
Verfasst              = {am}{Institut für Technische Informatik},
Titel auf Deutsch     = {
    Bewertung von RISC-V-Enklaven: Leistungsbenchmarks und bewährte Konfigurationspraktiken
}, 
Titel auf Englisch    = {
    Evaluating RISC-V Enclaves: Performance Benchmarks and Configuration Best Practices 
},
Autor                 = {Basil Ugbomoiko},
Betreuerin            = {Dr.-Ing. Saleh Mulhem},
Mit Unterstützung von = {Henrik Strunck, M.Sc},
Studiengang           = {IT Security},
Datum                 = {31. August 2025},
Abstract = {This thesis evaluates the performance of Keystone enclaves, an open-source Trusted Execution Environment (TEE) designed for RISC-V platforms. Keystone enables secure computation by isolating sensitive workloads from the rest of the system. To assess its performance, the study benchmarks critical system parameters, including the number of CPU cores, cache size, memory allocation, and enclave configuration. Using industry-standard benchmarking tools such as Dhrystone and CoreMark, the research measures execution time, context switch overhead, memory latency, and CPU utilization across a range of hardware and software configurations. The experimental analysis identifies key performance bottlenecks that impact the efficiency of enclave execution. Based on these findings, the study presents configuration best practices and tuning recommendations tailored to various workload profiles, such as compute-bound, memory-intensive, and mixed applications. The insights gained from this evaluation contribute to optimizing the deployment of Keystone enclaves, making them more viable for real-world use cases that demand secure and efficient execution on RISC-V architectures.},
Zusammenfassung = {Diese Arbeit bewertet die Leistung von Keystone-Enklaven, einer Open-Source Trusted Execution Environment (TEE) für RISC-V-Plattformen. Keystone ermöglicht sichere Berechnungen, indem sensible Workloads vom restlichen System isoliert werden. Zur Leistungsbewertung werden zentrale Systemparameter wie die Anzahl der CPU-Kerne, die Cache-Größe, die Speicherzuweisung und die Konfiguration der Enklave untersucht. Mithilfe standardisierter Benchmarking-Tools wie Dhrystone und CoreMark werden Ausführungszeit, Kontextwechsel-Overhead, Speicherlatenz und CPU-Auslastung unter verschiedenen Hardware- und Softwarekonfigurationen gemessen. Die experimentelle Analyse identifiziert wichtige Leistungsengpässe, die die Effizienz der Enklaven beeinträchtigen. Auf Basis dieser Erkenntnisse werden Konfigurationsrichtlinien und Optimierungsempfehlungen vorgestellt, die auf unterschiedliche Workload-Profile zugeschnitten sind, z. B. rechenintensive, speicherintensive und gemischte Anwendungen. Die gewonnenen Erkenntnisse tragen dazu bei, den Einsatz von Keystone-Enklaven zu optimieren und ihre praktische Anwendbarkeit in sicherheitskritischen Bereichen auf RISC-V-Architekturen zu verbessern.},
  Acknowledgements      = {
    This is the place where you can thank people and institutions, do
    not try to do this on the title page. The only exception is in
    case you wrote your thesis while working or staying at a company or abroad. Then you
},
Numerische Bibliographie
}